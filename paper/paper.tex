\documentclass{article}


\usepackage{arxiv}

\usepackage[utf8]{inputenc} % allow utf-8 input
\usepackage[T1]{fontenc}    % use 8-bit T1 fonts
\usepackage{hyperref}       % hyperlinks
\usepackage{url}            % simple URL typesetting
\usepackage{booktabs}       % professional-quality tables
\usepackage{nicefrac}       % compact symbols for 1/2, etc.
\usepackage{microtype}      % microtypography
\usepackage{fancyhdr}       % header
\usepackage{graphicx}       % graphics
\usepackage[table, x11names, svgnames]{xcolor}
\usepackage{subfig}
\usepackage{float}
\usepackage{caption}
\usepackage[section]{placeins}
\usepackage{setspace}
\usepackage{overpic}

\setstretch{1.2} % line spacing

\graphicspath{{assets/}}

%% Header
\pagestyle{fancy}
\thispagestyle{empty}
\rhead{ \textit{ }} 
\fancyhead[LO]{}

% \fancyhead[RE]{Firstauthor and Secondauthor} % Firstauthor et al. if more than 2 - must use \documentclass[twoside]{article}

%% Title
\title{Clinical advancement forecasting
%% Cite as
\thanks{\textit{\underline{Citation}}: 
\textbf{TBD}} 
}

%% Authors
\author{
  Authors TBD \\
  Related Sciences \\
%   %% examples of more authors
%    \And
%   Author3 \\
%   Affiliation \\
%   Univ \\
%   City\\
%   \texttt{email@email} \\
}

%% Body
\setlength{\parindent}{20pt}

\begin{document}
\maketitle

\begin{abstract}

Selecting optimal drug targets for diseases represents a critical juncture in the creation of new medicines. In this study, we present an automated framework for selecting such targets through a synthesis of human, animal, and cell model evidence that features simple, highly interpretable statistical models. These models are optimized based on the historical clinical success of target-disease pairings, use only 33 predictors derived directly from individual lines of evidence in Open Targets and can predict clinical advancement for those pairings with a success rate that is 1.5-2x higher than heuristic, composite scores derived from the same evidence as well as common measures of genetic support that have already been established -- and observed in this study -- to confer a 2x higher likelihood of success. We find that this cumulative 3-4x advantage is maximized when also utilizing prior human clinical validation specific to targets and diseases independent of pairings, or other static properties of targets like conservation, expression specificity and gene essentiality. These predictors are rarely absent, which contrasts highly with the evidence for target-disease pairs that rarely exists among those that have ever reached clinical trials. The extent and non-random nature with which this absence manifests pose key challenges in the selection of performance measures and predictor derivation. In particular, we address the possibility for reverse causality between clinical outcomes and evidence discovery through the use of longitudinal predictors. This novel aspect of our work also enables us to directly evaluate which types of evidence are most likely to leak information about clinical outcomes when not temporalized. By formulating target selection as a machine learning problem in this manner, and after accounting for these unique challenges, this framework can be easily deployed to prioritize disease targets yet to undergo clinical development. We explore some of these present-day targets, with a focus on those that are tractable, and offer all code and data used in this analysis as a proof-of-concept for our more comprehensive, proprietary ranking platform.
% Note: OT multiplier (2x) taken from Sensitivity / Relative risk ratio distribution across configurations / mean@10
\end{abstract}

\section{Introduction}

It has been well established that drugs with human genetic evidence linking their respective targets to indications in clinical trials are more likely to succeed \cite{Nelson2015-eg,King2019-rc,Minikel2023.06.23.23291765,Razuvayevskaya2023.02.07.23285407,PMID:30652614,PMID:24833294,PMID:35804044,PMID:37803084,PMID:36963162}, and to a lesser extent, that the same may also be true for single-cell transcriptomic evidence \cite{Dann2024.04.04.24305313}. This information has been used to devise many target and target-disease ranking algorithms based primarily on a synthesis of multiple genetic signals alone \cite{PMID:38172303,Koscielny2017-rr,PMID:31253980}. It is also possible to expand the breadth of this genetic support to more targets/diseases based on knowledge graphs, protein interactions and/or disease ontologies \cite{PMID:33262371,Bao2022-bq,Sadler2023-xd,PMID:36087372,PMID:36823319}. To our knowledge, all such expansion methods identify a larger space of opportunities at the expense of expected success rates. This is not a focus of this work as we aim, instead, to establish a framework for identifying targets/diseases with the very highest possible likelihood of success first. This is accomplished by integrating human clinical, genetic/genomic, transcriptomic and proteomic data as well as cell/animal model evidence, pathway information and basic literature metrics from Open Targets \cite{Koscielny2017-rr} in a simple statistical modeling framework. It can be contrasted with far more integrative methods that rely on neural and/or graph models over extensive knowledge graphs \cite{Paliwal2020-hr,PMID:33741907,pittala2020relationweighted,PMID:32750869}, which are more complex and difficult to interpret. We believe a desirable middle ground between these approaches and those that aim to combine many orthogonal indicators of success through expert knowledge in heuristic systems \cite{PMID:38404138,Koscielny2017-rr} would: 1) permit inclusion of many types of evidence from many sources, 2) be highly interpretable, 3) support expert judgement where necessary, and 4) not require manual ranking/weighting schemes.

A substantial challenge inherent to building such a system is the need to account for the longitudinal nature of knowledge discovery in biomedicine. This is vital because any method that optimizes for likely future clinical success based on historical clinical success may easily be biased by the non-random nature with which evidence is absent otherwise. We use only "temporalized" evidence, i.e. evidence for which timing of its emergence can be determined, and outcomes when training and evaluating our methods before ultimately applying them to present-day evidence with no restrictions on timing. We discuss motivations, prior research and our own analysis on how important this problem is for each source in Section~\ref{sec:results_inflation}.

Addressing such problems is common, but not ubiquitous, in studies that attempt to predict clinical trial outcomes using temporalized predictors \cite{PMID:37483175, PMID:34430930, Lo2019Machine}. The need for this is often clear in that setting where the inclusion of predictors like historical success rates for targets/diseases, trial sponsor track records, eventual patient enrollment, etc. constitute clear information leaks otherwise. This is discussed more in \cite{PMID:37483175} which notes several studies that do not account for this problem, and presents "quasi-prospective" as well as true "prospective" results. The difference between the two is that the former reconstructs timelines for predictors and outcomes based on recorded event dates while the latter relies on frozen predictions that are never evaluated until years later. Nomenclature for these formulations is conflicting though, where this definition of a "quasi-prospective" design is deemed entirely prospective in some cases, e.g. \cite{PMID:37225853}. We will refer to our design in this study as quasi-prospective since this definition is the best fit.

The prior works discussed so far can largely be categorized as either 1) target and target-disease prioritization methods evaluated based on how well they correlate with clinical trial success and 2) clinical trial outcome prediction models. Both are measured against the same outcomes and an important distinction between them lies in how the former methods are {\bf not} directly optimized for those outcomes while the latter methods are. In this study, we attempt to bridge these methodologies by predicting clinical trial advancement for target-disease pairs based solely on information that would be present well in advance of any drug program or individual trial. We then calibrate these predictions to determine what thresholds are necessary to match the observed success rates from benchmarks for genetic support like OMIM \cite{PMID:15608251}, ClinVar \cite{PMID:24234437} and GWAS. Finally, we examine how many present-day target-disease pairs are undeveloped (i.e. have never been in clinical trials), have a tractable target and are likely to see success rates matching or exceeding those calibrated benchmarks.

\section{Results}
\label{sec:results}

In order to model clinical advancement for target-disease (TD) pairs, we first define "advancement" as progression beyond any particular trial phase across all drugs associated with any one TD pair as indicated by the presence of a later-stage trial. All results to follow consider only advancement beyond phase 2 due to limitations described in Section~\ref{sec:discussion}. This binary outcome is then predicted based on a list of features shown in Table~\ref{tab:features}. Information for each of these features is only used when it was published before the year \textbf{prior} to the first phase 2 trial observed, with an exception for genetic evidence discussed in Section~\ref{sec:results_inflation}. A training dataset is then formed by including only TD pairs where this first phase 2 year is between 1990 and 2015. The evaluation dataset then consists of all TD pairs entering phase 2 between 2016 and 2022, with a 2 year offset from the present year (2024) to allow enough time for some trials to complete. While the average phase 2 trial duration may be as low as 2 years \cite{fdaStepClinical}, other estimates would suggest half of them take longer than 2.9 years \cite{PMID:29394327}. This means a substantial fraction of outcomes are censored, that this is an important parameter to test sensitivity to and that time itself is likely to be a crucial covariate in this formulation. The distribution of these outcomes, the number of associated targets/diseases and a variety of other statistics on this dataset are presented in Supplementary~Figure~\ref{fig:dataset_statistics}.

\subsection{Features}
\label{sec:features}

The features used throughout this study consist of 27 target-disease pair predictors, 5 target-specific predictors and 1 disease-specific predictor. These are listed in Table~\ref{tab:features}. The target and disease specific features are chosen carefully such that they are either capable of being associated with years in which events supporting them occurred or result from large-scale, unbiased methods that do not favor well-studied or drugged targets/diseases. Examples of this include target-specific tissue expression specificity scores computed from Human Protein Atlas \cite{PMID:25613900} and LOEUF \cite{PMID:32461654} scores from gnomAD. Simply put, our dataset combines scores from Open Targets for target-disease evidence and a select subset of target prioritisation \cite{OT23.12release} fields with almost no modifications, other than to add target and disease specific indicators of maximum trial phases reached and two extra genetic association features (described in Section~\ref{sec:methods}).

\begin{figure}[!htb]
  \centering
  \captionsetup{width=.9\linewidth}
  \captionsetup[subfigure]{labelformat=empty}
  \subfloat[\centering]{
    \begin{overpic}[width=.9\textwidth]{feature_presence_unscaled.pdf}
      \put(102, 20) {(a)}
    \end{overpic}
  }
  \qquad
  \subfloat[\centering]{
    \begin{overpic}[width=.9\textwidth]{feature_presence_scaled.pdf}
      \put(102, 20) {(b)}
    \end{overpic}
  }
  \caption{
    \textbf{Evaluation dataset feature presence}.
    (a) Presence and co-occurrence of various feature groupings or individual features indicated by overlapping sets of TD pairs, where the size of each bar is proportional to the number of TD pairs with a given combination of features and the sort ordering is determined by the number of TD pairs associated with any one feature (show in the right margin). The top margin shows the number of overlapping features for a combination of features and the bottom margin shows the number of TD pairs that are a part of that same combination.
    (b) Same as (a) without truly proportional bar sizes that are not scaled down beyond a limit necessary to fit all margin labels and combinatorial groupings. Features that are not specific to TD pairs are also omitted for brevity, along with the outcome feature used in this study and denoted here with the label "pair has advanced beyond phase 2".
  }
  \label{fig:feature_presence}
\end{figure}

The combination of target, disease and target-disease features creates sparsity patterns that are important to understand before interpreting models built from it. Figure~\ref{fig:feature_presence} demonstrates this sparsity by showing that of all the TD pairs entering phase 2 trials for the first time between 2016 and 2022 in our evaluation dataset (N=9010), less than 2\% of them ever have evidence directly linking them other than literature co-mentions, which exist for 21\% of those pairs. These TD pairs do, however, very frequently have prior clinical evidence for their associated targets and diseases. Specifically, 8,425 (94\%) have a target and 6,855 (76\%) have a disease that had already been in phase 2 or later trials previously. This is consistent with herding effects observed in recent drug development pipelines \cite{PMID:37117303} over the same time period (2016-2022) and underscores the prognostic value such information may have as it is becoming more and more common and clearly confers lower clinical risk for new drug programs. We also observe that, on top of the clear theoretical, causal relationship between prior target and disease human clinical validation and the likely success of programs for new combinations of such targets and diseases, these features have strong, univariate predictive effects in the evaluation dataset of our study. This is illustrated in Supplementary~Figure~\ref{fig:relative_risk_static_features}, which shows the relative risk of advancement for these features capturing the highest clinical stage previously reached by a target or disease.

Taken together, the paucity of TD pair evidence and the abundance of prior target or disease clinical validation should be considered carefully when interpreting predictive performance. TD pairs predicted to be highly likely to advance clinically despite a lack of target-disease-specific evidence are entirely plausible, but the value of such a prediction is dependent upon the application. We choose to minimize the influence of these cases in our study by focusing on rankings within therapeutic areas that do not extend beyond the number of TD pairs with direct evidence of some kind, as discussed in Section~\ref{sec:methods}. This choice is also reflected in our primary performance metrics, as discussed in Section~\ref{sec:metrics}.

\subsection{Models}
\label{sec:models}

We train a variety of models including constrained and unconstrained linear and tree models. The constrained variants of these models force effects of all features to increase monotonically, i.e. all effects are constrained to be non-negative. This is possible with no underlying feature transformations because all scores in Open Targets are constructed such that higher scores are presumed to be advantageous. 

We also apply these models to our evaluation dataset using several feature ablations in order to assess the value of groups of related features. We refer to a "core" feature set consisting of all features listed in Table~\ref{tab:features} except for the sole feature capturing the time since a target-disease pair first entered phase 2 trials (i.e. \colorbox{Gainsboro}{target\_disease\_\_time\_\_transition}). Combinations of learning algorithms for the models and the feature sets to which they are applied are referred to using the following convention:

\begin{itemize}
  \item \textbf{RDG}: Constrained, L2-regularized linear regressor (a.k.a. "Ridge regressor") fit with all core features
  \item \textbf{RDG-T}: \textbf{RDG} fit with all features instead of only core features, where the only difference is the inclusion of time since phase 2 transition for a TD pair
  \item \textbf{RDG-X}: \textbf{RDG} fit \textit{without} human clinical and genetic evidence features
  \item \textbf{GBM}: Constrained, gradient-boosted machine with fit with all core features
  \item \textbf{GBM-T}: \textbf{GBM} fit with all features instead of only core features
  \item \textbf{OTS}: Open Targets composite score
\end{itemize}

The omitted human clinical and genetic evidence features for the RDG-X model are all of those in Table~\ref{tab:features} with the midfix "clinical" or "genetic\_association". We omit these features specifically because they are known or expected to comprise good predictors of human clinical success, so their exclusion examines the extent to which only literature and animal model evidence along with target-specific properties accomplish this task.

In order to compare these models to an Open Targets composite score (OTS), we use an equally weighted sum of all scores from individual sources except for those assigned lower weights in \cite{OTweights}. Scores from these sources are multiplied by the corresponding weight before being summed and only the TD-specific features of Table~\ref{tab:features} are used. Neither the target/disease specific features nor the time since phase 2 transition feature are included in this calculation.

\subsection{Metrics}
\label{sec:metrics}

The primary performance metric used in this study is relative risk (RR). This metric is commonly used to assess univariate measures of genetic support \cite{Nelson2015-eg,King2019-rc,Minikel2023.06.23.23291765} and can be more intuitively understood, in the context of this study, as the probability that a TD pair among the top $N$ TD pairs as ranked by a particular method will advance beyond phase 2 trials divided by that same probability of advancement among TD pairs with a rank greater than $N$. This provides a means to compare multivariate, model-based methods to univariate methods on a common scale. More specifically, any RR metric reported for a model among top $N$ rankings is defined as:

\begin{equation}
  \frac{P(advancement | rank >= N)}{P(advancement | rank < N)}
\end{equation}

and RR metrics reported for univariate methods based on Open Targets scores for a single type of evidence, where not stated otherwise, are defined as:

\begin{equation}
  \frac{P(advancement | score > 0)}{P(advancement | score = 0)}.
\end{equation}

The use of such a metric is essential for properly assessing performance in this forecasting problem. While we also report more common measures of classifier performance like Receiver Operating Characteristic (ROC) and Average Precision (PR), neither of these adequately capture behavior in the upper extremes of rankings due the sparsity with which TD pair evidence is present for pairs that have ever entered phase 2 trials. This sparsity is further exacerbated in this study by the constraint that most of that evidence must have existed \textit{before} such trials began. For more details on the extent of this sparsity, see Section~\ref{sec:features}. The figure presented there, Figure~\ref{fig:feature_presence}, also demonstrates that a substantial fraction (78\%) of TD pairs that ultimately advance beyond phase 2 trials have targets and/or diseases with prior clinical validation despite no direct evidence linking the pairs themselves, which means a comprehensive measure of classifier performance (e.g. ROC) is far more likely to reflect the extent to which disease-only historical, clinical information or target-only information -- including other attributes like conservation, essentiality and tissue expression -- can predict clinical success. Again, this is not our primary focus as we want to evaluate the maximum achievable performance in this forecasting problem, and we assert that this is best accomplished when one or more lines of target-disease-specific evidence are present.

Reasonable alternative choices for this primary metric include those more common in information extraction literature or other machine learning studies with a focus on ranking rather than classification, such as mean reciprocal rank (MRR), precision at $k$ (P@k) and normalized discounted cumulative gain (NDCG) \cite{hoyt2022unified,moffat2022batch}. Precision at $k$ is the most similar among these to relative risk at $k$ since it is equivalent to the numerator in the relative risk calculation. We use relative risk instead because it is more intuitive than most ranking metrics, has well established analytical solutions for confidence intervals \cite{Katz1978-mo} and is consistent with prior work in this field.

Lastly, we emphasize that the interpretation of "risk" for the relative risk metric is to be inverted in this context. A higher "risk" in this study actually corresponds to a greater probability of success. The name "Relative Success" is used for this metric instead in \cite{Minikel2023.06.23.23291765} even though it has the same underlying definition. We choose not to use this label because we also present generic performance measures like ROC and AP, thereby prioritizing consistency with a domain-independent nomenclature.

\subsection{Performance}
\label{sec:results_performance}

\subsubsection{Open Targets comparison}

Figure~\ref{fig:performance_across_ta} demonstrates how well our primary model in this study, RDG, ranks TD pairs by comparison to a composite score from Open Targets, OTS. This comparison highlights relative risk (RR) as our primary performance indicator along with secondary measures of performance like Receiver Operating Characteristic (ROC) and Average Precision (PR), as discussed more in Section~\ref{sec:metrics}. The third ranking method presented in Figure~\ref{fig:performance_across_ta}, "RDG-T", differs from the RDG model only in that it uses time since the phase 2 transition as a predictive factor in addition to all others. We observe that the use of this information greatly improves standard performance metrics like receiver operating characteristic (ROC) and average precision (AP), however it adds little to no value in rankings beyond a level where substantial relative risk increases can be observed. In other words, it constitutes an effective but coarse mechanism for ranking TD pairs while lacking the high precision of other factors like genetic support. More implications of this and opportunities it may imply are discussed in Section~\ref{sec:discussion}. As a more practical concern, we refrain from focusing on RDG-T, or the similar GBM-T model, because neither is readily applicable to undeveloped TD pairs for which the time since phase 2 transition is not available. They do, however, present a useful performance ceiling towards which future work might build.

\begin{figure}[!htb]
  \centering
  \captionsetup{width=.9\linewidth}
  \captionsetup[subfigure]{labelformat=empty}
  \subfloat[\centering]{
    \begin{overpic}[width=.57\textwidth]{relative_risk_mean_across_ta.pdf}
      \put(50, -3) {(a)}
    \end{overpic}
  }
  \qquad
  \subfloat[\centering]{
    \begin{overpic}[width=.36\textwidth]{classifier_metrics_dist_across_ta.pdf}
      \put(27, -3) {(b)}
    \end{overpic}
  }
  \caption{
    \textbf{Performance compared to Open Targets composite scores}.
    (a) Equally-weighted average relative risk estimates across 13 therapeutic areas, by number of top rankings and 3 methods: RDG (ours), RDG-T (ours) and OTS (Open Targets composite scores). 
    (b) Receiver operating characteristic (ROC) and average precision scores across the same 13 therapeutic areas with no limit on the number of rankings. 
    See Supplementary~Figure~\ref{fig:relative_risk_by_ta} for raw data underlying (a).
  }
  \label{fig:performance_across_ta}
\end{figure}

We also note that Figure~\ref{fig:performance_across_ta} presents average RR estimates drawn across a subset of therapeutic areas, and the criteria used to select them is described more in Section~\ref{sec:methods}. A full list of therapeutic areas meeting these criteria can be seen in Supplementary~Figure~\ref{fig:relative_risk_by_ta} along with the RR estimates summarized in Figure~\ref{fig:performance_across_ta}. Furthermore, a comparison of the distribution of these estimates by model is presented in Supplementary~Figure~\ref{fig:relative_risk_dist_across_ta} along with the statistical significance of their differences.

\subsubsection{Genetic benchmark comparison}

In order to establish baseline levels of success and coverage across TD pairs, we examine ranking performance in comparison to well established, univariate indicators of genetic support in Figure~\ref{fig:relative_risk_by_limit}. This figure presents OMIM and GWAS baselines, in the parlance of \cite{King2019-rc}, \cite{Nelson2015-eg} and \cite{Minikel2023.06.23.23291765}, as well as an intermediate baseline from the European Variation Archive (EVA) \cite{PMID:34718739} containing evidence predominantly from ClinVar \cite{PMID:24234437}.

One key objective of this study is to determine if any model, e.g. RDG, can sort TD pairs with genetic support such that at least some portion of that sorted list has a likelihood of advancement that consistently exceeding what is expected from any one source of genetic support alone. We find that this goal is met and exceeded by the RDG model, which actually identifies more TD pairs than those that have either EVA or GWAS support alone at an expected rate of advancement exceeding that of the single source (respectively). This does not appear to be the case with the OMIM baseline, however the lack of examples in our evaluation dataset with OMIM support makes any determination difficult. See Section~\ref{sec:results_opportunities} for more on how these benchmarks are employed to contextualize opportunities among undeveloped TD pairs and Supplementary~Figure~\ref{fig:top_evaluation_predictions} for top predictions from the RDG model. This latter, supplementary figure further emphasizes our focus on prioritizing opportunities beyond those with genetic support and provides examples of TD pairs with multiple lines of evidence.

\begin{figure}[!htb]
  \centering
  \captionsetup{width=.9\linewidth}
  \includegraphics[width=1\textwidth]{relative_risk_by_limit.pdf}
  \caption{
    \textbf{Performance compared to genetic support benchmarks}. The RR estimates for each benchmark are based on the presence of the corresponding support across all TD pairs, and the number of pairs for which is present is represented by the horizontal bars extending horizontally from the y-axis. The bounds around the RDG RR estimates correspond to a Katz 90\% confidence interval.
  }
  \label{fig:relative_risk_by_limit}
\end{figure}

We also note that Supplementary~Figure~\ref{fig:relative_risk_core_features} shows confidence intervals for each of the genetic benchmarks of Figure~\ref{fig:relative_risk_by_limit} in isolation, as well as all other target-disease-specific evidence sources, in addition to confidence intervals for the RDG model at various top ranking cutoffs. Similar comparisons for target-specific and disease-specific features can be seen in Supplementary~Figure~\ref{fig:relative_risk_static_features}. These findings suggest that 1) genetic support for TD pairs is highly predictive but rare, 2) human clinical support for targets and diseases in isolation is also predictive while being more common and 3) constraint and expression specificity of targets exhibit modest but significant effects. While only human clinical evidence appears to have a prognostic value rivaling that of genetic support when considered on a univariate basis, the combined influence of non-genetic, non-clinical evidence is examined in Section~\ref{sec:model_comparison} where a model using only this information, RDG-X, still outperforms Open Targets composite scores by all measures.

\subsubsection{Model comparison}
\label{sec:model_comparison}

Figure~\ref{fig:performance_metric_mean_across_ta} presents average performance across therapeutic areas for select combinations of learning algorithm, constraint type and feature group described in Section~\ref{sec:models}. Several key findings illustrated in this figure are:

\begin{figure}[!htb]
  \centering
  \captionsetup{width=.9\linewidth}
  \begin{overpic}[width=1\textwidth]{performance_metric_mean_across_ta.pdf}
    \put(84.7, 38.2) {(a)}
    \put(84.7, 22.5) {(b)}
  \end{overpic}
  \caption{
    \textbf{Performance across model algorithms and feature ablation groups}.
    Average precision (AP) and receiver operating characteristic (ROC) scores with relative risk (RR) at ranking cutoffs denoted by \colorbox{Gainsboro}{RR@N}. 
    (a) Performance across constrained RDG models using different groups of features as described in Section~\ref{sec:models}
    (b) Performance across constrained (\colorbox{Gainsboro}{[+]}) and unconstrained (\colorbox{Gainsboro}{[+/-]}) linear and gradient-boosted models using the core feature set.
  }
  \label{fig:performance_metric_mean_across_ta}
\end{figure}

\begin{enumerate}
  \item The RDG-T model achieves far higher ROC and AP scores through the use of the time since transition feature, which indicates the number of years a TD pair has been underdevelopment after having reached phase 2.
  \item The RDG model, however, matches or exceeds RDG-T in performance among top TD pairs
  \item The RDG-X model, using no human clinical or genetic evidence linked to a disease, outperforms the Open Targets composite score and nearly matches the performance of the RDG model beyond top rankings
  \item Linear models outperform gradient-boosting models by nearly all measures
  \item Constrained linear models outperform unconstrained linear models by nearly all measures
\end{enumerate}


We conclude from these results that constrained linear models are an optimal choice for this problem due both to their greater performance and interpretability. This interpretability is illustrated more in Section~\ref{sec:effects} and owed much to the effort Open Targets has already undertaken to construct evidence scores such that they can be assumed to have a monotonically increasing effect on the likelihood that a causal relationship exists between a target and a disease.


\subsection{Effects}
\label{sec:effects}

The coefficients learned by the RDG model, and the average effects they have across the evaluation dataset, are shown in Figure~\ref{fig:effect_sizes}. This model most highly prioritizes genetic signals that have the greatest coverage, i.e. associations from GWAS studies through the \colorbox{Gainsboro}{ot\_genetics\_portal} feature and associations from any curated clinical genetics source, i.e. EVA, Orphanet, UniProt, Genomics England, ClinGen and gene2phenotype, via the \colorbox{Gainsboro}{curated} \vspace*{0mm} feature.  Notably, literature and target/disease specific clinical features also have substantial effects, followed by indicators of animal evidence and target genetic constraint / expression specificity. Any features not shown were deflated to have no effect, which is possible in this model due to the non-negativity constraint. One such feature worth emphasizing is transcriptomic evidence from Expression Atlas. We found this somewhat surprising, but it is supported by arguments against transcript over/under expression as an indicator of genes that influence disease rather than the other way around \cite{PMID:34561431}.

\begin{figure}[!htb]
	\centering
  \captionsetup{width=.9\linewidth}
	\includegraphics[width=1\textwidth]{effect_sizes.pdf}
  \caption{
    \textbf{RDG model feature effects}. The \textbf{effect max} values are equivalent to RDG model coefficients for the corresponding feature while the \textbf{effect mean} values indicate average values of the product between the coefficient and a particular feature value, when that feature is present.
  }
	\label{fig:effect_sizes}
\end{figure}

It is worth noting that the discordance between the coefficients and the average feature effects of Figure~\ref{fig:effect_sizes} arises from both the frequency with which features exist and the distribution of their underlying scores. Scores for many clinical genetics features (e.g. OMIM, Genomics England, UniProt) are very frequently absent or close to 1. By comparison, scores for literature associations are typically far lower, even when limited only to cases where they exist, with a median value of 0.12 (mean=.23) in the evaluation data. This is why the \colorbox{Gainsboro}{europepmc} feature has a relatively large associated coefficient and a much smaller average effect on predictions.


\subsection{Opportunities}
\label{sec:results_opportunities}

A common method for identifying druggable opportunities within a specific disease context involves first ranking TD pairs according to some prioritization methodology followed by filtering or reprioritizing those ranks based on knowledge of target tractability \cite{PMID:28356508,PMID:35401535,PMID:31253980}. We use a similar approach to identify tractable targets associated with TD pairs that have yet to enter clinical trials. To aid in interpreting this approach, we also draw on the results of Figure~\ref{fig:relative_risk_by_limit}. The data in this figure suggests thresholds for the RDG model that align to expected rates of advancement compared to several genetic support benchmarks. These thresholds are used to bucket undeveloped TD pairs before further bucketing them based on levels of tractability. The tractability buckets in Supplementary~Table~\ref{tab:tractability_buckets} provide \colorbox{Gainsboro}{HIGH}, \colorbox{Gainsboro}{MED}, and \colorbox{Gainsboro}{LOW} confidence ratings for each type of tractability evidence based on the priorities suggested in \cite{OTTractability}.

\begin{figure}[!htb]
  \centering
  \captionsetup{width=.9\linewidth}
  \includegraphics[width=1\textwidth]{opportunity_summary.png}
  \caption{
    \textbf{Present-day target-disease pair counts by stage, likelihood of advancement and tractability}. The \textbf{stage} panel contains counts by maximum trial phase reached, the \textbf{threshold} panel contains counts of pairs with a RDG model score exceeding that of the associated benchmark for only undeveloped pairs, and the \textbf{tractability} panel shows pair frequencies among undeveloped pairs exceeding the EVA threshold that also have a HIGH tractability rating as defined in Supplementary~Table~\ref{tab:tractability_buckets}. This corresponds to targets that have all been in clinical development already, except for the \textbf{OC} modality in which case it indicates that a target has been approved. 
  }
  \label{fig:opportunity_summary}
\end{figure}

Figure~\ref{fig:opportunity_summary} shows the distribution of TD pair counts for select buckets across therapeutic areas as well as across the current maximum phase reached for any one pair. We find that there are $\sim$2,400 small-molecule-enabled, $\sim$1,400 antibody-enabled, and 14 PROTAC-enabled TD pairs with a probability of advancement that is nearly 3x other TD pairs based on the EVA threshold RR=2.93 in Figure~\ref{fig:relative_risk_by_limit}. Top antibody-enabled pairs are shown in Supplementary~Figure~\ref{fig:top_opportunity_predictions} along with their corresponding genetic and clinical support.

\subsection{Inflation}
\label{sec:results_inflation}

Like most studies of this kind, we assume a "closed-world" \cite{Paliwal2020-hr} over the space of target-disease pairs and any evidence between them. This means that we do not differentiate between evidence that an association for any one pair truly does \textbf{not} exist (or is too weak to be relevant under the omnigenic model \cite{PMID:28622505}), and the lack of any attempt to find that evidence in the first place. This also means that our estimate of the prognostic value for any one evidence source is subject to historical trends in biomedical research and the myriad ways that this research can be biased towards particular targets and diseases. We avoid attempting to comprehensively survey these biases in favor of offering an illustrative list of specific examples that are relevant in this study:

\begin{enumerate}
\item Mendelian randomization research is biased towards cardiovascular diseases as they have a disproportionate number of known, modifiable exposures \cite{PMID:36736292}
\item Putative protein interactions that do not result from genome-scale or otherwise unbiased assays result in an overrepresentation of successful drug targets in resources like STRING \cite{PMID:36370105}, thereby inflating the success of network expansion methods over these databases to identify such targets \cite{Sadler2023-xd}.
\item Transcript expression studies run in late-stage clinical trials for a single indication, e.g. \cite{PMID:27723281} linking SLE to IFN genes, are a degenerate indicator of advancement beyond earlier stage trials when the timing of this evidence is not accounted for.
\item Targets tested against more indications in clinical trials enrich for failures because the marginal cost of testing more indications decreases, but the evidence for these indications is often weaker \cite{PMID:33262371}.
\item Herding effects in pharma R\&D pipelines around particular drug targets are becoming increasingly clear over time \cite{PMID:37117303} and generate an excess of clinical evidence for those targets.
\end{enumerate}

We also note that the skew in basic drug target research towards those that already have rich annotations and well characterized molecular function \cite{PMID:29358745} as well as the disproportionate representation of particular target families in pharma R\&D pipelines \cite{PMID:27910877,PPR:PPR7029} and the fact that literature is well known to be biased away from negative results in general \cite{PMID:32893970} are all problematic. 

While it is not possible to address all of these issues, we emphasize that there is a clear pattern across the examples in the list above in that they require \textbf{past} clinical successes and/or failures to arise in the first place. This suggests that accounting for when evidence first emerged would limit the extent of these problems. We do so in this study based solely on publication dates associated with any one piece of information linking target-disease pairs. This also offers a novel opportunity to attempt to quantify what kind of evidence suffers most from these biases. Figure~\ref{fig:evidence_inflation} presents results for this based on a relative risk statistic defined as:

\begin{equation}
  \frac{P(A | B)}{P(A | \neg B)}
\end{equation}

where:

\begin{itemize}
\item \(A\) is the event that evidence for a TD pair arises after its first early-stage (phase 1 or 2) trial rather than before
\item \(B\) is the event that a TD pair advances into late-stage trials (phase 3 or 4)
\end{itemize}

We refer to this as "inflation risk" so as not to confuse it with the relative risk statistic used in all other contexts, and it can be more simply described as the fraction of TD pairs for which evidence arises \textbf{after} the beginning of an ultimately successful early-stage trial divided by that same fraction for TD pairs that do not advance to late-stage trials. The intuition for this statistic is that it will be higher if successful trials lead to the generation of evidence of a particular type, and it should be 1 in cases where the emergence of evidence is independent of clinical success. We also measure this potential lack of independence through the more commonly used Fisher's exact test, e.g. \cite{PMID:19725948}, and both are presented in Figure~\ref{fig:evidence_inflation}.

\begin{figure}[!htb]
  \centering
  \captionsetup{width=.9\linewidth}
  \includegraphics[width=1\textwidth]{evidence_inflation.pdf}
  \caption{
    \textbf{Clinical success drives evidence discovery}.  
  }
  \label{fig:evidence_inflation}
\end{figure}

We find that evidence from Reactome is the worst offender by this metric, implying that it often only arises for TD pairs after a certain level of clinical success has been attained. We also find that long-running aggregators/curators of published research often focused on individual diseases/phenotypes, like Expression Atlas, IMPC, CGS and Cancer Biomarkers exhibit this form of inflation as well.

Sources of genetic evidence appear to be much less inflated, or have too little data to reach significance. This is to be expected for GWAS evidence arising from genome-wide, phenome-wide biobank consortia, however much of historical GWAS evidence is not phenome-wide. More context on how much this is likely to matter comes from \cite{PMID:37612393} in which it was estimated that as few as 6\% of 500 FDA-approved targets for non-cancer drugs arose from programs highly motivated by pre-existing genetic support and that "the remaining 94\% were probably identified using conventional pharmacology, biochemistry or molecular biology approaches". We then speculate that if the initiation of new drug programs was not historically motivated highly by the existence of genetic support, then the incentives for pursing new genetic evidence based on clinical and commercial success are likely to be minimized. This, in conjunction with existing precedent \cite{Nelson2015-eg,King2019-rc,Minikel2023.06.23.23291765,Razuvayevskaya2023.02.07.23285407,PMID:30652614,PMID:35804044} and our inflation results, ultimately led us to the use of genetic evidence without temporalization. In other words, we do not treat genetic evidence as longitudinal features like all others associated with TD pairs. A breakdown of which features are treated in which manner is provided in Table~\ref{tab:features}.

\section{Conclusion}

We have demonstrated that simple machine learning methods applied to longitudinal biomedical evidence from many sources can be used to predict clinical outcomes for combinations of drug targets and diseases, without knowledge of molecular properties or trial design details. We have also shown that these methods are more precise in the extremes of their predictions than composite, heuristic scores like those from Open Targets. They also outperform such baselines by more comprehensive, traditional measures of classifier performance; however, we find this less compelling and easier to accomplish than improving performance among the upper tail of the opportunities implied by the very highest predictions. This framework would also support the addition of new lines of evidence over time well as it is designed to automatically determine the relevance of any new information without intervention.  Lastly, we find that the space of present-day, undeveloped targets within a disease context that both exceed baseline levels of tractability and have a high predicted likelihood of clinical advancement is substantial. It is likely to grow quickly as well since the breadth of much of the underlying evidence is expanding rapidly \cite{PMID:33214558,PMID:36634672,PMID:31491408}.

\section{Discussion}
\label{sec:discussion}

\begin{itemize}
  \item Cover limitations with OT concerning temporalization for both evidence and drug approvals, and why only transitions from phase 2 are relevant for this work
  \item Discuss the possibility to do prospective evaluation with OT snapshots
  \item Discuss the limitations of the tractability analysis and why they might not be tractable targets
  \item Discuss why we present metrics like AP and ROC at all, and what kinds of use cases might maximize for something like ROC instead
  \item Expand on how the inclusion of time greatly improves tail performance and mention that this is consistent with its nature as a necessary but not sufficient condition for success, and it is likely that much of the value it adds in the tail of lower rankings could be captured and enhanced if other early indicators for the many reasons trials fail \cite{Razuvayevskaya2023.02.07.23285407} were also included (see more on this in Section~\ref{sec:discussion})
  \item We are focusing on prioritizing among targets/indications with genetic support, rather than expanding this space to find opportunities with weaker support
  \item "There is evidence that a 9.6\% vs. 13.8\% success rate for drugs from phase 1 trials to approval may mean a \$480 million difference in the median research and development cost required to bring a new drug to the market (Wouters, McKee, and Luyten 2020)." \cite{PMID:34930919}
  \item From \cite{PMID:33262371}: "It is important to bear in mind therefore that what we are measuring when looking at historical trial outcomes is not an unbiased measure of any given gene's true disease associations, but rather a view on how useful a given evidence source or analytical method has been for choosing drug targets based on current and historical drug discovery practices. Dramatic changes in these practices in the future could render some of our conclusions obsolete, though the fundamental observation that genetic association itself is retained in molecular networks will remain valid."
  \item Add select tractability and DepMap features?
\end{itemize}

\section{Methods}
\label{sec:methods}

\begin{itemize}
  \item Discuss why all target prioritisation data fields are not used due to the potential leakage they may impose (e.g. target families, GO annotations, etc.)
  \item Describe how therapeutic areas were selected based on having at least 100 TD pairs with target-disease-specific evidence of any kind and with explicit omissions: ("biological\_process", "pregnancy or perinatal disease", "injury, poisoning or other complication", "pregnancy or perinatal disease", "medical procedure", "infectious disease", "animal disease")
  \item RDG models are fit using the \colorbox{Gainsboro}{Ridge} implementation from scikit-learn and GBM models are fit using the LightGBM \cite{LightGBM} algorithm
  \item Cite supervenn
  \item Training/evaluation results are limited by temporalization while the present-day predictions are not
  \item OMIM is defined as EVA associations with publications
  \item The "genetic\_association\_\_curated" field is a union of all genetic association sources other than "gene\_burden" and "ot\_genetics\_portal"
  \item Mention that the scores for a TD pairs in a year are the maximum score for that source, not the harmonic sum
  \item Mean imputation is used for target-specific features
  \item Mention specifics on RDG and GBM implementations, i.e. lightGBM and scikit-learn
  \item RDG is always trained on all features, but only applied to feature subsets where relevant
\end{itemize} 

\section{Appendix}

\subsection{Features}

\input{assets/feature_table.tex}

\begin{figure}[!htb]
  \centering
  \captionsetup{width=.9\linewidth}
  \includegraphics[width=1\textwidth]{dataset_statistics.png}
  \caption{
    \textbf{Training and evaluation dataset summary statistics}.
  }
  \label{fig:dataset_statistics}
\end{figure}


\section{Supplementary Material}

\subsection{Performance}

\begin{figure}[H]
  \centering
  \captionsetup{width=.9\linewidth}
  \includegraphics[width=1\textwidth]{relative_risk_core_features.pdf}
  \caption{
    \textbf{Performance of individual features and predictive scores as measured by relative risk}.
    RDG model results denoted by \colorbox{Gainsboro}{rdg@N} indicate performance for the \colorbox{Gainsboro}{N} top TD pairs. The same convention is used for \colorbox{Gainsboro}{literature} evidence and the \colorbox{Gainsboro}{time\_\_transition\_\_eq\_X} convention denotes RR estimates when the time since the phase 2 transition is equal to \colorbox{Gainsboro}{X} years. The \colorbox{Gainsboro}{omim}, \colorbox{Gainsboro}{eva}, and \colorbox{Gainsboro}{ot\_genetics\_portal} features correspond to the OMIM, EVA and OTG baselines of Figure~\ref{fig:relative_risk_by_limit}, respectively. All other features are assessed based on their existence. The counts along the origin indicate how many TD pairs were used to compute the RR numerator.
  }
  \label{fig:relative_risk_core_features}
\end{figure}

\begin{figure}[H]
  \centering
  \captionsetup{width=.9\linewidth}
  \includegraphics[width=1\textwidth]{relative_risk_static_features.pdf}
  \caption{
    \textbf{Relative risk scores for target/disease features}.
    The features ending with \colorbox{Gainsboro}{qtl\_Q} denote binary indicators constructed from cases where the feature meets or exceeds quantile \colorbox{Gainsboro}{Q} of its distribution. The features ending with \colorbox{Gainsboro}{gte\_X} denote indicators for when the feature meets or exceeds a specific value \colorbox{Gainsboro}{X}.
  }
  \label{fig:relative_risk_static_features}
\end{figure}

\begin{figure}[H]
  \centering
  \captionsetup{width=.9\linewidth}
  \includegraphics[width=1\textwidth]{relative_risk_by_ta.png}
  \caption{
    \textbf{Relative risk scores by method, benchmark and therapeutic area}.
    The \textbf{average} therapeutic area indicates mean values across all others except for \textbf{all}, which is an ungrouped estimate across all diseases regardless of therapeutic area.
  }
  \label{fig:relative_risk_by_ta}
\end{figure}
  
\begin{figure}[H]
  \centering
  \captionsetup{width=.9\linewidth}
  \includegraphics[width=1\textwidth]{relative_risk_dist_across_ta.pdf}
  \caption{
    \textbf{Relative risk distributions across select therapeutic areas}.
    P-values are computed from a one-sided Wilcoxon signed-rank test with the alternative that the RDG model RR averages across therapeutic areas exceed OTS averages.
  }
  \label{fig:relative_risk_dist_across_ta}
\end{figure}

\begin{figure}[H]
  \centering
  \captionsetup{width=.9\linewidth}
  \includegraphics[width=1\textwidth]{relative_risk_model_features.pdf}
  \caption{
    \textbf{Performance by algorithm, constraint type and feature grouping}.  
  }
  \label{fig:relative_risk_model_features}
\end{figure}


\subsection{Predictions}

\begin{figure}[H]
  \centering
  \captionsetup{width=.9\linewidth}
  \includegraphics[width=1\textwidth]{top_evaluation_predictions.png}
  \caption{
    \textbf{Top RDG model evaluation dataset predictions}.
    Feature contributions are shown as the product of their underlying values and the RDG coefficients. The \textbf{advanced} field indicates whether the associated TD pair advanced beyond phase 2 as of 2024.
  }
  \label{fig:top_evaluation_predictions}
\end{figure}


\subsection{Opportunities}

\begin{table}
\centering
\caption{Tractability bucket assignments}
\label{tab:tractability_buckets}
\begin{tabular}{llll}
\toprule
 & evidence & modality & confidence \\
\midrule
1 & Phase 1 Clinical & OC & LOW \\
2 & Advanced Clinical & OC & MED \\
3 & Approved Drug & OC & HIGH \\
4 & GO CC med conf & AB & LOW \\
5 & Human Protein Atlas loc & AB & LOW \\
6 & UniProt SigP or TMHMM & AB & LOW \\
7 & UniProt loc med conf & AB & LOW \\
8 & GO CC high conf & AB & MED \\
9 & UniProt loc high conf & AB & MED \\
10 & Advanced Clinical & AB & HIGH \\
11 & Approved Drug & AB & HIGH \\
12 & Phase 1 Clinical & AB & HIGH \\
13 & Database Ubiquitination & PR & LOW \\
14 & Half-life Data & PR & LOW \\
15 & Small Molecule Binder & PR & LOW \\
16 & Literature & PR & MED \\
17 & UniProt Ubiquitination & PR & MED \\
18 & Advanced Clinical & PR & HIGH \\
19 & Phase 1 Clinical & PR & HIGH \\
20 & Druggable Family & SM & LOW \\
21 & High-Quality Pocket & SM & LOW \\
22 & Med-Quality Pocket & SM & LOW \\
23 & High-Quality Ligand & SM & MED \\
24 & Structure with Ligand & SM & MED \\
25 & Advanced Clinical & SM & HIGH \\
26 & Approved Drug & SM & HIGH \\
27 & Phase 1 Clinical & SM & HIGH \\
\bottomrule
\end{tabular}
\end{table}


\begin{figure}[H]
  \centering
  \captionsetup{width=.9\linewidth}
  \includegraphics[width=.9\textwidth]{top_opportunity_predictions.png}
  \caption{
    \textbf{Top ranked undeveloped, tractable target-disease pairs}. The highest scoring TD pairs per the RDG model that have not entered clinical trials despite having a target that has been in trials of an antibody-based drug. All values shown other than \textbf{prediction} are raw feature values, unweighted by RDG coefficients.
  }
  \label{fig:top_opportunity_predictions}
\end{figure}


\subsection{Sensitivity}

In order to validate the stability of our findings in Section~\ref{sec:results_performance}, we repeat this analysis across 18 different configurations listed in Supplementary~Table~\ref{tab:sensitivity_configurations}. This includes 3 separate versions of Open Targets, 3 choices for the year defining the split between training and evaluation data and 2 choices for the length of the minimum advancement window (in years).

We find that the mean RR values from the RDG model consistently exceed the OTS model in all configurations among the very highest rankings (N=10) and also exceed the OTS model in all configurations except for 1 for N between 20 and 60. This data is shown in Supplementary~Figure~\ref{fig:sensitivity_relative_risk}. The significance of these differences drops notably after N=40, which can be seen in the distribution of p-values from a Wilcoxon signed-rank test shown in Supplementary~Figure~\ref{fig:sensitivity_p_values}.

\begin{table}
\centering
\caption{Configurations for sensitivity analysis}
\label{tab:sensitivity_configurations}
\begin{tabular}{llrr}
\toprule
 & open\_targets\_version & max\_training\_year & min\_time\_to\_advancement\_years \\
\midrule
1 & 23.09 & 2017 & 4 \\
2 & 23.12 & 2017 & 2 \\
3 & 23.09 & 2015 & 2 \\
4 & 23.12 & 2015 & 2 \\
5 & 23.09 & 2017 & 2 \\
6 & 23.06 & 2015 & 4 \\
7 & 23.06 & 2017 & 2 \\
8 & 23.06 & 2013 & 2 \\
9 & 23.06 & 2015 & 2 \\
10 & 23.06 & 2017 & 4 \\
11 & 23.12 & 2013 & 4 \\
12 & 23.09 & 2015 & 4 \\
13 & 23.09 & 2013 & 2 \\
14 & 23.12 & 2013 & 2 \\
15 & 23.09 & 2013 & 4 \\
16 & 23.12 & 2017 & 4 \\
17 & 23.12 & 2015 & 4 \\
18 & 23.06 & 2013 & 4 \\
\bottomrule
\end{tabular}
\end{table}


\begin{figure}[H]
  \centering
  \captionsetup{width=.9\linewidth}
  \includegraphics[width=1\textwidth]{sensitivity_relative_risk.pdf}
  \caption{
    \textbf{Relative risk distributions across configurations in sensitivity analysis}.
    The distribution of the mean RR values displayed for a single configuration in Figure~\ref{fig:performance_across_ta} is shown here across 18 configurations.
  }
  \label{fig:sensitivity_relative_risk}
\end{figure}

\begin{figure}[H]
  \centering
  \captionsetup{width=.9\linewidth}
  \includegraphics[width=1\textwidth]{sensitivity_p_values.pdf}
  \caption{
    \textbf{P-value distributions across configurations in sensitivity analysis}.
    The distribution of the p-values displayed for a single configuration in Supplementary~Figure~\ref{fig:relative_risk_dist_across_ta} is shown here across 18 configurations.
  }
  \label{fig:sensitivity_p_values}
\end{figure}


\bibliographystyle{unsrt}  
\bibliography{references} 

\end{document}
